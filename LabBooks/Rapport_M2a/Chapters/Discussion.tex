% Chapter Template

\chapter{Discussion et perspectives} % Main chapter title

\label{Discussion} % For referencing this chapter elsewhere, use \ref{Discussion}

%----------------------------------------------------------------------------------------
Notre modèle semble ainsi capable de suivre le fonctionnement d'un modèle POMDP en réalisant à tour de rôle une observation de son environnement et une action ayant pour objectif d'améliorer la perception de cet environnement. \\
L'agent va donc réaliser une série de saccades jusqu'à réussir à placer la cible visuelle au niveau de sa fovéa, où sa description (ici sa catégorisation) pourra être réalisée avec la plus grande accuité possible, et donc avec le plus grand taux de réussite. Ce comportement `'atteindre-et-décrire`' instauré par le fonctionnement POMDP semble cohérent avec ce que l'on peut observer dans les systèmes biologiques \autocite{Werner2014, Najemnik2005}. \\
En complément à ces observations sur le comportement générale du modèle, nous avons pu observer que plus une cible est éloignée de la fovéa de l'agent lorsqu'elle est présentée initialement, moins ses prédictions seront précises et en conséquences plus nombreuses seront les saccades destinées à atteindre la position de la cible. 
Encore une fois, ce comportement semble cohérent avec ce que l'on observe dans les systèmes biologiques. 
A noter toutefois que le profil de performances du modèle, notamment concernant l'évolution de la taille de ses erreurs de prédiction avec l'excentricité de la cible, correspond à de faibles performances biologiques. 
Plus exactement, le modèle semble présenter un profil de performances similaire au système biologique lorsque la cible est présentée peu de temps (150ms), mais lorsqu'elle est présentée plus longtemps (1s) les performances du système biologique ne semblent pas être influencées par l'excentricité de la cible. Des comparaisons quantitatives seront ici nécessaires pour confirmer ou infirmer ces similarités \autocite{Uddin2004}.\\

% Prospects
	% Modèle probabiliste
	% Modif entrainement  utiliser classifieur pour réaliser apprentissage détecteur
	% Bruit écologique
	% Implémentation robotique
	
Plusieurs étapes ont dors et déjà été identifiées afin de rendre le modèle à la fois plus performant et plus proche d'une certaine réalité neurologique. \\
La première consistera en la modification de la prédiction du modèle (à l'heure actuelle, la prédiction correspond à deux coordonnées `'certaines`' où la cible devrait être présente) en prédiction probabiliste. 
Cette transition permettrait de traiter la perception du modèle comme une carte de probabilité (ou de chaleur) où chaque point de l'espace est relié à une probabilité de contenir la cible. Ainsi la prédiction ne sera pas réalisée sur un point précis de l'espace mais sur un ensemble de points dont l'ecart-type devrait augmenter avec l'excentricité (d'après les résultats observés sur la figure~\ref{fig:err_distance}). De plus, cette carte de probabilité se mettant à jour à chaque nouvelle saccade (puisqu'une nouvelle observation de l'environnement est alors réalisée), le problème de recherche de la localisation précise de la cible devrait se résoudre de lui-même en explorant tour à tour chacune des localisations les plus probables \autocite{Butko2010, Najemnik2005}.\\

Une seconde étape sera de reconstruire entièrement les méthodes d'apprentissage afin de non seulement augmenter fortement les performances du \textit{classifieur}, pour l'instant loin d'approcher les performances standards des modèles \textit{machine learning} de vision artificielle, mais aussi de tenter d'augmenter au maximum les performances du \textit{détecteur} pour tenter d'atteindre les performances biologiques. \\
L'une des solutions envisageables pour ce dernier point consiste en la transition d'un apprentissage indépendant des deux couches, comme c'est le cas actuellement, vers un modèle d'apprentissage utilisant la sortie (sous la forme d'une carte de probabilité) du \textit{classifieur} pour guider l'entraînement (c'est à dire s'en servir comme d'un label) du \textit{détecteur}. Cette optique permettrait non seulement d'améliorer les performances du modèle, mais aussi d'augmenter son autonomie vis à vis de l'utilisateur humain en produisant lui-même une partie de ses labels d'apprentissage. \\

Lorsque le modèle présentera des performances jugées suffisantes, l'étape suivante sera probablement de le mettre à l'épreuve dans des conditions pour lesquelles il n'aura pas été programmé afin de tester la robustesse de la généralisation obtenue grâce à l'apprentissage automatisé. Pour cela, l'un des moyens à notre disposition est d'altérer la qualité de l'image utilisée en entrée via l'introduction de bruit. 
Idéalement, ce bruit devra être construit pour suivre des règles écologiques, c'est à dire qu'il devra modéliser le bruit que l'on peut retrouver lors d'une perception naturelle de l'environnement visuel. Deux méthodes sont pour le moment envisagées : introduire des pseudo-stimulis créés en `'mélangeant`' des stimuli utilisés pour l'apprentissage (permettant d'obtenir des régions comprenant les mêmes variances de niveaux de gris que lors de la présence d'un stimulus réel mais sans présenter d'information pertinente) et/ou créer un bruit corrélé par transformation mathématiques \autocite{Najemnik2005}. \\

Finalement, lorsque le modèle aura été entraîné, évalué et optimisé dans un ensemble de contextes artificiels, il deviendra envisageable de soumettre son fonctionnement à des contextes écologiques en l'implémentant dans un système autonome.
Le paradigme POMDP permettra à l'agent de choisir l'action la plus adaptée en fonction de sa perception de son environnement et de la tâche à réaliser, tandis la présence de l'apprentissage automatisé et de l'approche neuromimétique (vision rétinienne) permettra de conserver une performance élevée (notamment en adaptabilité et en vitesse de réaction) tout en diminuant au maximum la puissance de calcul et la consommation d'énergie nécessaires à son fonctionnement. \autocite{Potthast2016} \\
En prenant pour example un agent robotisé mobile, tel qu'un drone, dont la tâche serait de reconnaitre un objet ou un visage dans son environnement visuel, ce modèle pourrait lui permettre de choisir de façon active et autonome quelle action choisir (réaliser une saccade et si oui en visant quelle position, réaliser un mouvement et si oui lequel parmi 6 directions et 4 rotations disponibles) suivant son contexte environnemental et les informations qui lui sont accessibles (position estimée de l'agent par rapport à la cible, taux de certitude de l'identification, environnement observé, etc) afin de réaliser sa tâche de façon optimale. \autocite{Potthast2016} \\