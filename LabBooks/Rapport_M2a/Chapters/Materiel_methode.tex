% Chapter Template

\chapter{Matériel et méthodes} % Main chapter title
\label{Materiel_methode} % Change X to a consecutive number; for referencing this chapter elsewhere, use \ref{ChapterX}

%----------------------------------------------------------------------------------------

\section{Support physique} %Spécificités de l'ordinateur, spécificités de la machine virtuelle.
L'ensemble des modélisations ont été réalisés sur une ordinateur personnel hébergeant une machine virtuelle :\\

\resizebox{18cm}{!}{
\begin{tabular}{| p{4cm} || l | l | p{5cm} | l | l |}
\hline
& ID & Système d'explotation & Processeur & Mémoire vive & Carte graphique\\ \hline
Machine physique & ASUS ROG G75VW & Windows 7 64-bit SP1 & Intel Core I7-3610QM 2,30GHz (8CPU) &  
8GB DDR3 & NVIDIA GeForce GTX670M\\ \hline
Machine virtuelle (ressources allouées) & VirtualBox v.5.2.6 & Ubuntu 16.04 & 4 CPU, 90\% des ressources & 
5298 Mo & Le support GPU n'a pas été utilisé\\ \hline
\end{tabular}
}

%----------------------------------------------------------------------------------------

\section{Support numérique} %Programmes, version de python, librairies utilisées.

%----------------------------------------------------------------------------------------

\section{Modèle POMDP} %Description du modèle perception-action, schéma explicatif

%----------------------------------------------------------------------------------------

\section{Régression linéaire multivariée} %Formules du modèle mathématiques soutenant l'apprentissage