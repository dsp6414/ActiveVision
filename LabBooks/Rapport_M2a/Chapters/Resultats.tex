% Chapter Template

\chapter{Résultats} % Main chapter title
 
\label{Résultats} % Change X to a consecutive number; for referencing this chapter elsewhere, use \ref{ChapterX}

%----------------------------------------------------------------------------------------

Cette partie contient des résultats préliminaires, le modèle étant encore en cours de construction et d'optimisation lors de l'écriture de ce rapport.

\section{Apprentissage supervisé}

L'étape de benchmarking du paramètre d'apprentissage $\alpha$ (équation~\ref{eqn:Grad_desc}) permet de rendre compte de son importance sur l'efficacité de l'apprentissage et du modèle. \\
Dans le cadre du filtre \textit{Wavelets} et lorsque seulement le \textit{détecteur} est entraîné, on peut ainsi observer qu'une valeur $\alpha\geq0.4$ entraîne un sur-apprentissage très important (le coût augmente fortement au cours de l'entraînement, figures~\ref{fig:benchmark_surApp1} et \ref{fig:benchmark_surApp2}), tandis que $\alpha=0.3$ semble représenter une valeur optimale pour l'apprentissage (figure~\ref{fig:benchmark_alpha}).\\
Par contre, lorsque \textit{détecteur} et \textit{classifieur} sont tous deux entraînés, deux jeux de poids indépendants doivent être optimisés par l'apprentissage. Chaque couche possède ainsi son propre paramètre $\alpha$ (respectivement $\alpha_{detect}$ et $\alpha_{classif}$) et l'on peut donc calculer leurs coûts indépendamments (figure~\ref{fig:logpolar_cost}).

%----------------------------------------------------------------------------------------

\section{Prédiction de la position}

Après avoir été entraîné, le modèle semble capable de détecter la cible dans son environnement et de prédire précisemment sa position (figure~\ref{fig:saccades_wavelets} et \ref{fig:saccades_logpolar}). Il semble aussi capable d'utiliser cette prédiction pour réaliser une saccade aux coordonnées prédites de la cible visuelle, ce qui modifie en conséquence sa perception de l'environnement.\\
Mais une seule saccade n'est pas toujours suffisante pour atteindre la cible (figure~\ref{fig:sacc_nombre}) et le nombre de saccades nécessaires augmente avec la distance initiale de la cible du centre de fixation (figure~\ref{fig:sacc_distance}). Cette relation pourrait provenir de la diminution de l'acuité avec l'excentricité dans la champs visuel (provoquée par le champs rétinien), entraînant une diminution de la précision des prédictions (figure~\ref{fig:err_distance}).\\
Ainsi, plus une cible est éloignée de la fovéa lorsqu'elle est présentée initialement, moins les prédictions du modèle seront précises et en conséquences plus nombreuses seront les saccades destinées à l'atteindre.

%----------------------------------------------------------------------------------------

\section{Prédiction de la catégorie}

%----------------------------------------------------------------------------------------
