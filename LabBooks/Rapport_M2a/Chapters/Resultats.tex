% Chapter Template

\chapter{Résultats} % Main chapter title
 
\label{Résultats} % Change X to a consecutive number; for referencing this chapter elsewhere, use \ref{ChapterX}

%----------------------------------------------------------------------------------------

Cette partie contient des résultats préliminaires, le modèle étant encore en cours de développement et d'optimisation lors de l'écriture de ce rapport. Les idées émises dans ce chapitre ainsi que le suivant restent pour le moment des hypothèses qui devront nécessairement être confirmées lors de travaux ultérieurs. \\

\section{Apprentissage supervisé}

L'étude d'étalonnage du paramètre d'apprentissage $\alpha$ (équation~\ref{eqn:Grad_desc}) permet de rendre compte de son importance sur l'efficacité de l'apprentissage et du modèle. On peut ainsi observer que certaines valeurs entraînent un sur-apprentissage très important (figures~\ref{fig:benchmark_surApp2}), tandis que d'autres semblent représenter des valeurs utilisables, voire optimales, pour réaliser l'apprentissage (figure~\ref{fig:benchmark_alpha}).\\
Lorsque \textit{détecteur} et \textit{classifieur} sont tous deux entraînés, deux jeux de poids indépendants doivent être optimisés par l'apprentissage. Chaque couche possède ainsi son propre paramètre $\alpha$ (respectivement $\alpha_{detect}$ et $\alpha_{classif}$) et l'on peut donc calculer leurs coûts indépendamments (figure~\ref{fig:logpolar_cost}).\\
On peut observer qu'indépendamment du filtre utilisé et du nombre de couches entraînées, l'apprentissage par descente de gradient permet d'optimiser le modèle, en modifiant les poids $\theta$ itération après itération. Cette optimisation est révélée par une diminution graduelle du coût, représentant une différence entre la réalité et ce qui est prédit par le modèle.

%----------------------------------------------------------------------------------------

\section{Prédiction de la position}

Après avoir été entraîné, le modèle semble capable de détecter la cible dans son environnement visuel et de prédire précisemment sa position dans l'espace (figure~\ref{fig:saccades_wavelets} et \ref{fig:saccades_logpolar}). L'agent est ensuite capable d'utiliser ces connaissances pour réaliser une saccade jusqu'aux coordonnées prédites de la cible visuelle, ce qui modifie en conséquence sa perception de l'environnement et donc de la cible.\\

L'agent doit parfois réaliser plus d'une saccade pour atteindre la cible (figure~\ref{fig:sacc_nombre}). On observe un nombre important d'essais où une ou deux saccades sont suffisantes pour atteindre sa position réelle, puis un nombre décroissant d'essais pour un nombre de saccades supérieure à 3. \\
Le nombre de saccades pour atteindre la cible semble dépendre de la distance à laquelle elle se trouve lorsqu'elle est présentée pour la première fois à l'agent (figure~\ref{fig:sacc_distance}). On observe que cette relation semble suivre une loi linéaire croissante jusqu'à un premier seuil (pour une distance initiale d'environ 25 pixels) après lequel on observe un plateau du nombre moyen de saccades. On peut ensuite observer un second seuil (pour une distance initiale d'environ 35 pixels) où le nombre moyen de saccades augmente à nouveau fortement. A noter que l'agent ne réalise pas de saccade lorsque la cible est directement présentée dans sa fovéa.\\
Cette relation entre le nombre de saccades nécessaires pour placer la cible dans la fovéa et la distance initiale à laquelle est présentée la-dite cible pourrait dépendre de la taille des erreurs que va commettre l'agent lorsqu'il va prédire la position de la cible (figure~\ref{fig:err_distance}). On observe en effet que cette erreur tend à croitre linéairement avec la distance à laquelle se trouve la cible lors de la prédiction. On retrouve ici l'un des seuils de la figure précédente (pour une distance d'environ 35 pixels) où l'erreur de prédiction augmente fortement et en rupture avec la croissance qui avait lieu jusque là. Cette augmentation de la taille de l'erreur peut se traduire par une diminution de la précision des prédictions de la position de la cible. Plus la cible est éloignée de sa fovéa, plus l'agent semble imprécis.\\

%----------------------------------------------------------------------------------------

%\section{Prédiction de la catégorie}

%----------------------------------------------------------------------------------------
