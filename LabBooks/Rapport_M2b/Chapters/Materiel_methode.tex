% Chapter Template

\chapter{Matériel et méthodes} % Main chapter title
\label{Materiel_methode} % Change X to a consecutive number; for referencing this chapter elsewhere, use \ref{ChapterX}

%--------------- Matériel ---------------
\section{Matériel}
L'ensemble des simulations (comprenant apprentissage et évaluation) ont été réalisées sur une machine connectée à distance via un protocole ssh et dont les caractéristiques sont visibles dans la~\autoref{tab:materiel}.	\\

%--------------- Filtre LogPolar ---------------
\section{Filtre LogPolaire}
Afin de simuler une variabilité de l'acuité visuelle chez notre modèle, nous avons utilisé un filtre LogPolaire (figure~\ref{fig:logpol_filter}).
Ce filtre, construit avec une approche neuromimétique, est constitué d'un ensemble de filtres Gabor et vise à reproduire la forme et l'organisation réelle des champs récepteurs présents dans les régions visuelles des systèmes nerveux biologiques. De précédentes études ont montré qu'il présente un certain nombre d'avantages pour la modélisation des systèmes biologiques, notamment car il est aisément modifiable pour simuler les champs récepteurs de différentes régions impliquées dans la vision (rétine, corps genouillé latéral, colliculis supérieur, V1 puis aires associatives).
Le filtre LogPolaire correspond en réalité à une matrice de valeurs qui, lorsque appliquée à une image (par multiplication matricielle) permet une décroissante de la résolution en fonction de l'excentricité (la distance par rapport au centre de l'image). 
Le résultat de l'application de ce filtre sur l'un de nos stimuli est visible sur la figure~\ref{fig:mnist_128_LP}. \autocite{Freeman2011} \\
Une version de ce filtre dans laquelle les filtres Gabor d'un même emplacement (mais ne possédant pas la même orientation) sont moyennés est appliquée à la carte de certitude, servant de \textit{label} pour l'apprentissage de notre modèle (figures~\ref{fig:energy_filter} et \ref{fig:accuracy_128_LP}). \\

%--------------- Bruit écologique ---------------
\section{Bruit écologique}
Pour permettre à nos stimuli de s'approcher de ceux pouvant être reçus et traités par les systèmes biologiques, nous avons superposé à nos signaux un bruit généré de manière aléatoire et selon deux méthodes possibles.
La première consiste en la génération de bruit Perlin \autocite{Perlin1985} (figure~\ref{fig:perlin_noise}), permettant à l'origine de produire automatiquement des textures à l'aspect naturel (nuages, feu, eau, etc).
La seconde consiste en la génération de bruit \textit{MotionCloud} (figure~\ref{fig:motioncloud_noise}), permettant d'obtenir des textures aléatoires et naturelles servant à l'origine à étudier la perception des mouvements (!!! INSERT REF !!!) \\

%--------------- Catégorisation ---------------

%--------------- Localisation ---------------